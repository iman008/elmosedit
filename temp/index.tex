\hypertarget{german-tcb-theorem}{%
\subsection{German tcb-theorem}\label{german-tcb-theorem}}

Definition: Titel

Eine Folge in einer beliebigen Menge \textbackslash(X\textbackslash) ist
eine Funktion \textbackslash{[} a:\textbackslash mathbb\{N\}
\textbackslash rightarrow X, \textbackslash quad (a\_1,a\_2,a\_3,...) =
(a\_n)\_\{n \textbackslash geq 1\} = (a\_n)\_n = (a\_n)
\textbackslash{]} Man spricht von Zahlenfolgen,
\textbackslash(\textbackslash dots\textbackslash)

\textbackslash(x\^{}2=\textbackslash frac\{asdasd-asdas\textbackslash sqrt\{x\}\}\{ssa\}\textbackslash)
\$\$ \textbackslash begin\{align*\} AP \textbackslash times AX\&=AK
\textbackslash times AM \textbackslash\textbackslash{}
\&=AN\textbackslash times AY \textbackslash end\{align*\} \$\$

تعریف 1.2

تابع اختلاف قوت نقطه \$P\$ نسبت به دو دایره \$\textbackslash omega\_1\$
و \$\textbackslash omega\_2\$ را به شکل زیر تعریف میکنیم:
\textbackslash{[}\textbackslash mathbf\{P\}(P,\textbackslash omega\_1,\textbackslash omega\_2)=Pow\_\{\textbackslash omega\_1\}\^{}P
-Pow\_\{\textbackslash omega\_2\}\^{}P\textbackslash{]}

قضیه 1.2

فرض کنید که دوایر \$\textbackslash omega\_1\$ و
\$\textbackslash omega\_2\$ به مراکز \$O\_1\$ و \$O\_2\$ در صفحه هستند و
خط \$l\$ محور اصلی این دو دایره است و نقطه \$P\$ نقطه ای در صفحه
\$\textbackslash omega\_1\$ و \$\textbackslash omega\_2\$ است. آنوقت
رابطه زیر برقرار است:
\textbackslash{[}\textbackslash mathbf\{P\}(P,\textbackslash omega\_1,\textbackslash omega\_2)=2dist(P,l)
\textbackslash overline\{O\_1O\_2\} \textbackslash{]}

قضیه 1.2

ابتدا فرض میکنیم که \$P\$ و \$N\$ به ترتیب وسط \$AB\$ و \$AC\$ هستند.
طبق فرض مسئله و متوازی الاضلاع بودن \$PNMB\$ داریم: \$\$
\textbackslash angle KNM=\textbackslash angle PBM=\textbackslash angle
KYM \$\$ پس نتیجه میگیریم که \$KNYM\$ محاطی هست به نحو مشابه نتیجه میشود
که \$KPXM\$ نیز محاطی است. طبق قوت نقطه داریم: \$\$
\textbackslash begin\{align*\} AP \textbackslash times AX\&=AK
\textbackslash times AM \textbackslash\textbackslash{}
\&=AN\textbackslash times AY \textbackslash end\{align*\} \$\$ پس
\$PNYX\$ نیز محاطی است. این نتیجه میدهد که \$PN\$ و \$XY\$ پادموازی اند
و چون \$PN\textbackslash mid \textbackslash mid BC\$ پس \$XY\$ و \$BC\$
نیز پادموازی اند بنابراین \$BXYC\$ نیز محاطی است و مسئله اثبات میشود
\textbackslash{[} \textbackslash begin\{align*\}
\textbackslash mathbf\{P\}(M,\textbackslash omega\textquotesingle,\textbackslash omega)\&=(\textbackslash frac12)\textbackslash mathbf\{P\}(S,\textbackslash omega\textquotesingle,\textbackslash omega)+(\textbackslash frac12)\textbackslash mathbf\{P\}(T,\textbackslash omega\textquotesingle,\textbackslash omega)\textbackslash\textbackslash{}
\textbackslash implies
Pow\_M\^{}\{\textbackslash omega\textquotesingle\}\&=\textbackslash frac\{AS\^{}2+AT\^{}2-ST\^{}2\}\{4\}\textbackslash\textbackslash{}
\&= \textbackslash frac\{2 AS \textbackslash times AT
\textbackslash cos\{(\textbackslash angle
A)\}\}\{4\}\textbackslash\textbackslash{}
\&=\textbackslash frac\{AP\^{}2\}\{2\} \textbackslash end\{align*\}
\textbackslash{]}

مثال 1.2

ابتدا فرض میکنیم که \$P\$ و \$N\$ به ترتیب وسط \$AB\$ و \$AC\$ هستند.
طبق فرض مسئله و متوازی الاضلاع بودن \$PNMB\$ داریم: \$\$
\textbackslash angle KNM=\textbackslash angle PBM=\textbackslash angle
KYM \$\$ پس نتیجه میگیریم که \$KNYM\$ محاطی هست به نحو مشابه نتیجه میشود
که \$KPXM\$ نیز محاطی است. طبق قوت نقطه داریم: \$\$
\textbackslash begin\{align*\} AP \textbackslash times AX\&=AK
\textbackslash times AM \textbackslash\textbackslash{}
\&=AN\textbackslash times AY \textbackslash end\{align*\} \$\$ پس
\$PNYX\$ نیز محاطی است. این نتیجه میدهد که \$PN\$ و \$XY\$ پادموازی اند
و چون \$PN\textbackslash mid \textbackslash mid BC\$ پس \$XY\$ و \$BC\$
نیز پادموازی اند بنابراین \$BXYC\$ نیز محاطی است و مسئله اثبات میشود
\textbackslash{[} \textbackslash begin\{align*\}
\textbackslash mathbf\{P\}(M,\textbackslash omega\textquotesingle,\textbackslash omega)\&=(\textbackslash frac12)\textbackslash mathbf\{P\}(S,\textbackslash omega\textquotesingle,\textbackslash omega)+(\textbackslash frac12)\textbackslash mathbf\{P\}(T,\textbackslash omega\textquotesingle,\textbackslash omega)\textbackslash\textbackslash{}
\textbackslash implies
Pow\_M\^{}\{\textbackslash omega\textquotesingle\}\&=\textbackslash frac\{AS\^{}2+AT\^{}2-ST\^{}2\}\{4\}\textbackslash\textbackslash{}
\&= \textbackslash frac\{2 AS \textbackslash times AT
\textbackslash cos\{(\textbackslash angle
A)\}\}\{4\}\textbackslash\textbackslash{}
\&=\textbackslash frac\{AP\^{}2\}\{2\} \textbackslash end\{align*\}
\textbackslash{]}

{\$\$ E = mc\^{}2 \$\$}
